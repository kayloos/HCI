\documentclass[a4paper]{article}

\usepackage[utf8]{inputenc}
\usepackage[T1]{fontenc}
\usepackage[bitstream-charter]{mathdesign}

\usepackage{graphicx}
\usepackage{booktabs}
\usepackage{subfig}

\usepackage[danish]{babel}
\renewcommand{\danishhyphenmins}{22} % bedre orddeling

\usepackage{fancyhdr}
\pagestyle{fancy}
\fancyhead[L]{Menneske-datamaskine Interaktion\\ Gruppe 61}
\fancyhead[R]{2011/09/13\\ Øvelse \#2}

\begin{document}

\section{Arbejdsopgaver}

Valget af arbejdsopgaver er lavet med blik på at få os rundt på de sider en
gennemsnitlig bruger ville besøge. Arbejdsopgaver skal indeholde opgaver som vi 
(uden at have set siden) forventer man kan udføre på siden.\\
Vi har derfor valgt at udført følgende arbejdsopgaver:

\begin{itemize}
\item{\textbf{Opgave 1:}}
    Undersøg prisen for en Ford Transit fra København til Fyn
\item{\textbf{Opgave 2:}}
    Bestil en varevogn fra København til Fyn
\item{\textbf{Opgave 3:}}
    Ændre returpunktet fra Fyn til Jylland
\item{\textbf{Opgave 4:}}
    Slet bestillingen
\end{itemize}

\section{Elementer i brugergrænsefladen}
\subsection{Fokus på brugerens opgaver}
De har en brugercentreret forside, fordi de har valgt at have
reservationsformularen på forsiden. Det hovedsalige formål for brugere af siden,
er at reservere/leje biler, og derfor giver det god mening at give netop denne
funktionalitet meget synlighed. I det aspekt er der fokus på brugerens opgaver
(Figur \ref{forside}).

\begin{figure}[htbp]
  \begin{center}
    \includegraphics[scale=.6]{1.png}
  \end{center}
  \caption{Forside, reservationsformular.}
  \label{forside}
\end{figure}

En anden opgave en bruger kan have, er muligheden for at annullere en ordre. På
dette punkt har siden gemt funktionaliteten væk i en underside, der ikke giver
mening Figur \ref{annullering}.

\begin{figure}[htbp]
  \begin{center}
    \includegraphics[scale=.6]{3.png}
  \end{center}
  \caption{Annullering af ordre gemt væk.}
  \label{annullering}
\end{figure}

\subsection{Synlighed af systemets status}
Systemets status er relevant når siden henter informationer til brugeren. Når
applikationen finder ledige biler frem, vises systemets status mens der bliver
hentet information. Det opfylder kravet om synlighed Figur \ref{status}.

\begin{figure}[htbp]
  \begin{center}
    \includegraphics[scale=.6]{4.png}
  \end{center}
  \caption{Systemets status.}
  \label{status}
\end{figure}

\subsection{Sammenhæng mellem systemet og den virkelige verden}
Siden bruger ikke komplicerede eller tekniske termer til at beskrive bilerne man
kan leje, men holder derimod sproget i menigmandszonen (Figur \ref{sprog}).

\begin{figure}[htbp]
  \begin{center}
    \includegraphics[scale=.6]{5.png}
  \end{center}
  \caption{Sproget er i menigmandszonen.}
  \label{sprog}
\end{figure}

Siden viser menupunkter og information i en naturlig orden, hvor det vigtigste
for brugeren bliver vist først. Se Figur \ref{menu}.

\begin{figure}[htbp]
  \begin{center}
    \includegraphics[width=\textwidth]{2.png}
  \end{center}
  \caption{Menupunkter.}
  \label{menu}
\end{figure}

\subsection{Brugerens kontrol og frihed}
Der er god mulighed for at hoppe frem og tilbage i forskellige stadier af
reservationen. Det er også muligt at slette reservationen når den er blevet
oprettet (Figur \ref{stadier}).

Stadieoversigten kan dog være svær at få øje på.
\begin{figure}[htbp]
  \begin{center}
    \includegraphics[width=\textwidth]{9.png}
  \end{center}
  \caption{Forskellige stadier af bestillingsprocessen.}
  \label{stadier}
\end{figure}

\subsection{Konsistens og standarder}
Vi har ikke været i stand til at finde handlinger, ord eller situationer der
betyder det samme og forvirrer brugerne.

\subsection{Fejl-udbedring}
Hvis man indtaster en afleveringsdato der er tidligere end afhentningsdatoen,
kommer siden med to fejlmeddelelser, hvilket ikke er videre elegant (se Figur
\ref{fejl_datoer}).

\begin{figure}[htbp]
  \begin{center}
    \includegraphics[scale=.6]{6.png}
  \end{center}
  \caption{Fejlmeddelelse ved forkert indtastning af datoer.}
  \label{fejl_datoer}
\end{figure}

Når man indtaster kontaktinformation forkert, viser siden de fejlagtige
indtastninger korrekt (Figur \ref{fejl_boks} og \ref{fejl_kontaktinformation})

\begin{figure}[htbp]
  \begin{center}
    \includegraphics[scale=.6]{10.png}
  \end{center}
  \caption{Fejlboks ved forkert indtastning af kontaktinformation.}
  \label{fejl_boks}
\end{figure}

\begin{figure}[htbp]
  \begin{center}
    \includegraphics[scale=.6]{8.png}
  \end{center}
  \caption{Fejl ved manglende eller fejlagtig kontaktinformation}
  \label{fejl_kontaktinformation}
\end{figure}


\subsection{Brugerinput huskes}
Siden husker indtastede data på tværs af stadier. Brugeren bliver ikke tvunget
til selv at huske noget.

Der er hjælpeknapper, og de handlinger hvis betydning ikke er indlysende har en
forklaring

\subsection{Fleksibilitet og effektivitet}
Der er ikke mulighed for at lagre sin kontaktinformation på siden, ved fx at
oprette en bruger.

Det er heller ikke muligt at bruge en tidligere reservering. Man kunne
forestille sig en bruger, der jævnligt lejer den samme bil, fra det samme
afhentningssted og afleverer den ved det samme afleveringssted.

\subsection{Æstetik og minimalistisk design}
De holder et stabilt farveskema (læs: orange og blåt) igennem alle sidens
undersider og funktioner. Siden er ikke videre køn, og der bliver brugt meget
plads på forsiden til reklame for firmaet, hvilket ikke er minimalistisk.

I bestillingsproceduren er reklamerne fjernet, og derfor opfylder de kravene om
et minimalistisk design, selv om sidens design ligner noget fra et andet
årtusinde.

\subsection{Forebyggelse af fejl}
De har små i'er udenfor alle tekstbokse som brugeren skal udfylde. Disse i'er (Figure \ref{help_kontaktinformation})
giver information om hvad der skal stå i boksen og hjælper derved brugeren med at
undegå fejl.

\begin{figure}[htbp]
  \begin{center}
    \includegraphics[scale=.6]{7.png}
  \end{center}
  \caption{Hjælp til korrekt indtastning af information.}
  \label{help_kontaktinformation}
\end{figure}

\subsection{Hjælp og dokumentation}
Siden har en række shortcuts nederst på siden der kan hjælpe brugeren med at
navigere. Dog ville en søgefunktion hvor brugeren selv kunne skrive søgekriterierne
kunne hjælpe brugere. hvis man ser på Figur \ref{menu} kan man se alle menuens 
punkter, men der ikke er ikke et åbenlyst `hjælp`-punkt. Dette er selvfølgeligt 
heller ikke så nødvendigt når det er en hjemmeside, og ikke et alenestående 
program der har sine egne genvejs-taster og indstillinger. Hvis man kunne oprette
en bruger, og siden huskede på ens informationer og farvoritter, ville et dette
være mere nødvendigt.

\section{Nytteværdi, effektivitet og tilfredshed}
Siden opretholder kravet om nytteværdi. Alle væsentlige funktioner som er krævet
af brugeren er opfyldt.

Effektiviteten for siden er høj for en ny bruger, mens der ikke er meget at
hente, hvis man er en tilbagevendende bruger, der ofte bruger sidens
funktionalitet. Man kan ikke gemme sine kontaktoplysninger over længere tid, og
man kan heller ikke automatisk gentage en reservation.

Om brugeren er tilfreds eller ej, afhænger af om hans opgave bliver løst. Siden
gør det let for brugeren at løse den åbenlyse opgave, (reservér eller lej en
bil), og det er også den funktion der er lagt mest vægt på. Hvis brugerens
opgave er at annullere en bestilling, afhænger brugerens tilfredshed om han er i
stand til at finde stedet hvor man kan annullere, hvilket ikke er sandsynligt.

\section{Diskussion}

Retningslinjerne er nyttige på den måde, at de fungerer som en opskrift. Man kan
eksaminere en funktion, vurdere om retningslinjerne er opfyldt, og så hurtigt 
finde frem til hvor der skal laves ændringer. 

Det retningslinjerne ikke hjælper med, er hvilke ændringer der skal laves. 
Det vil være usandsynligt at finde en generel opskrift på løsningsforslag, 
fordi man skal tage højde for den måde applikationen er udformet på. Et 
løsningsforslag skal indeholde konkrete eksempler på hvordan løsningen
kan udformes og være klar nok til ikke at foresage misforståelse og 
fejlede løsningsforsøg.

\end{document}
