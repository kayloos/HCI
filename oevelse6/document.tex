% This is a template of the report proposed used for the fourth mandatory exercise on the HCI course at DIKU 2011.
% The template is an approximately direct translation of the original report produced and published by Rolf Molich, www.dialogdesign.dk
% The template is only partial - only selected portions of the original template is included to show it "works". The rest can be synthesized.
% The template probably contains some redundant definitions or other stuff - Remove, update, improve as you see fit.
%
% This file comes in a zip file called ex6-template-tex.zip. The content of the zip file is:
% ex6-template.tex - The main file.
% ex6.sty - A style file for the frontpage and definitions. Must be included on the tex search path/same dir as ex6-template.tex.
% Pics/ - A directory containing 8 smallish PNG files used for compiling this file.
%
% The template was created by Casper Petersen.

\documentclass[10pt,a4paper]{article}      % Book.cls is also usable
\usepackage[utf8]{inputenc}                % UTF8 encoding
\usepackage[T1]{fontenc}                   % Default fonttype and style
\usepackage[danish]{babel}                 % Danish hyphenation pattern
\usepackage[bitstream-charter]{mathdesign}
\usepackage[small,margin=1cm]{caption}
\usepackage{graphicx}                      % For graphics
\usepackage{subfig}                        % For subfigures - Can be removed and replaced with standard figures
%% \usepackage{a4wide}                        % Gives us a bit extra spaces in the margins - Can be removed.
\usepackage{color, colortbl}               % Use to define colors and give tables a colored background
\usepackage{fancyhdr}                      % Fancy headers? Yes please.
\usepackage{ex6}                           % Separate form and function plus ugly tex code
\usepackage{url}
\usepackage{booktabs}

\author{Michael Budde, Kasper Passov, Niels Ørbæk Christensen, Claus Skou Nielsen}
\title{Test af brugervenlighed af www.faarupsommerland.dk}
\renewcommand{\customer}{Fårup Sommerland}
\date{Oktober 2011}

\usepackage{enumitem}
\setitemize{
    itemsep=0pt
}
\setenumerate{
    itemsep=0pt
}
\setdescription{
  font=\normalfont\bfseries,
  style=sameline,
  leftmargin=\parindent,
  itemsep=0pt,
  listparindent=\parindent
}
\newlist{opgaver}{enumerate}{2}
\setlist[opgaver,1]{label=\arabic*.}
\setlist[opgaver,2]{label=\alph*.}

\newlist{kommentarer}{itemize}{1}
\setlist[kommentarer]{
    leftmargin=0.15\textwidth,
    rightmargin=0.1\textwidth,
    itemsep=3mm
}
\setlength\fboxsep{0pt}
\setlength\fboxrule{.3pt}
\definecolor{gray}{gray}{0.6}

\fancyhf{}
\fancyhead[L]{Test af Fårup Sommerlands websted}
\fancyhead[R]{Oktober 2011}
\fancyfoot[C]{\thepage}
\pagestyle{fancy}
%\renewcommand{\headrulewidth}{0.4pt}
%\renewcommand{\footrulewidth}{0.4pt}

\newcommand\pic[1]{\includegraphics[trim=0 6 -10 0]{Pics/#1}}
\renewcommand\good{\pic{good}}
\renewcommand\goodidea{\pic{goodidea}}
\renewcommand\smallproblem{\pic{smallproblem}}
\renewcommand\seriousproblem{\pic{seriousproblem}}
\renewcommand\criticalproblem{\pic{criticalproblem}}
\renewcommand\filler{\pic{filler}}


\begin{document}

\maketitle
\newpage
\setcounter{page}{1}

\section*{Resumé}
\setcounter{page}{1} % sets the current page number to 32
\addcontentsline{toc}{section}{Resumé}

Faarup Sommerland præsenterer og tilbyder information om forlystelsesparken på
\url{faarupsommerland.dk}. Vi, de studerende, har udført en brugertest af
hjemmesiden, med udgangspunkt i oplægget fra Faarup Sommerland.

%Kolding Kommune tilbyder information om kommunen på www.koldingkom.dk.
%Webstedet blev lanceret i december 2002. DialogDesign  har gennemført en test
%af brugervenlighed af webstedet den 12. og 13-03-2003.

%Formålet med testen er at finde og beskrive problemer i dialogen
%mellem typiske brugere og Kolding Kommunes websted, samt at foreslå
%velgennemprøvede forbedringsforslag som løser de påpegede problemer.

%Testen er foretaget ved at bede otte personer om at løse typiske
%opgaver på webstedet under kyndig overvågning. Denne rapport beskriver de
%positive forhold og de problemer som testen har afsløret.

\noindent De væsentligste punkter hvor webstedet fungerer godt:
\begin{itemize}
  \item \textbf{Indholdet er godt}\\ Testdeltagerne var imponerede og overraskede over de store mængder nyttigt indhold på webstedet. Sproget er godt, og testdeltagerne fandt ingen sproglige fejl.
  \item \textbf{Navigation}\\ Navigationen fungerer i det store og hele godt. Testdeltagerne udnyttede effektivt samspillet mellem menuen i venstre side og Mere om boksene i højre side.
  \item \textbf{Svartiderne er gode}
\end{itemize}

\noindent De væsentligste punkter hvor webstedet kan fungere endnu bedre:
\begin{itemize}
  \item \textbf{Søgefunktionen}\\
  Anbefalinger til forbedring af søgefunktionen:
  \begin{itemize}
    \item Find de 100-200 søgeord som brugerne synes er vigtigst, ved at analysere søgeord. Giv disse søgeord særbehandling, så de altid fører til et ``perfekt'' søgeresultat.
    \item Giv en konstruktiv meddelelse hvis en søgning ikke giver nogen resultater.
    \item Udelad protokoller, mødereferater, intranetsider og lignende fra standard-søgningen.
    \item Vis den sammenhæng hvori søgeordet optræder, for hvert søgeresultat. Fremhæv søgeordet.
  \end{itemize}
  \item \textbf{Integration med Netborger.dk}\\
  Netborger skal integreres bedre med Kolding Kommunes websted. Se afsnit 9.
  \item \textbf{Effektivitet}\\
  Gør genveje og indeks mere synlige. Tilbyd en kort introduktion til webstedet. Se afsnit 3.
\end{itemize}

\clearpage

\tableofcontents
\clearpage


\section{Kategorier af kommentarer}
Testdeltagernes kommentarer er klassificeret i følgende kategorier:


\begin{table}[!ht]
\centering
\rule{\linewidth}{\heavyrulewidth}\\[6mm]
\begin{kommentarer}

\item[\good] \textbf{Godt}

Denne måde at gøre tingene på syntes testdeltagerne godt om. Den kan tjene som forbillede for
andre.

\item[\goodidea] \textbf{God idé}

Et forslag fra en testdeltager eller testlederen, som kan medføre en væsentlig forbedring af
brugeroplevelsen.

\item[\smallproblem] \textbf{Mindre problem}

Testdeltagerne studsede et kort øjeblik.

\item[\seriousproblem] \textbf{Alvorligt problem}

Problemet forsinkede testdeltagerne i 1-5 minutter, men testdeltagerne kom videre af sig selv.
Gav lejlighedsvis anledning til katastrofer.

\item[\criticalproblem] \textbf{Kritisk problem}

Gav anledning til hyppige katastrofer. En katastrofe er en situation, hvor webstedet ``vandt''
over testdeltagerne, dvs. en situation som forhindrede testdeltagerne i at løse en rimelig
arbejdsopgave på webstedet, eller som irriterede testdeltagerne voldsomt.

\end{kommentarer}
\rule{\linewidth}{\heavyrulewidth}
\caption{Kategori symboler anvendt i denne rapport}
\label{tab:gt}
\end{table}%
\clearpage



\section{Navigation og opdeling}
\begin{kommentarer}

\item[\good]{\textbf{Synligheden af menuen er god}}

All testdeltagere fandt menuen med det samme.

\item[\seriousproblem]{\textbf{Dårlig navngivning af menupunkterne}}

Testdeltagerne havde besvær med at regne ud hvad de forskellige
menupunkter dækker over hvilket førte til at de ofte måtte alle siderne igennem for at finde
det de ledte efter. Specielt menupunkterne \emph{Parken}, \emph{Planlæg dit besøg} og \emph{Information} skabte forvirring.

\item[\goodidea] \textbf{Mouseover-dropdown menupunkter}

En af deltagerne foreslog at der skulle komme en dropdown-menu når man førte musen hen over menupunkterne, der viste hvilke undermenupunkter der var inde under siden. Dette kunne give et bedre overblik, så man ikke skulle klikke ind på alle siderne for at finde et bestemt underpunkt.

\item[\seriousproblem]{\textbf{Link til søgesiden er for usynligt}}

1 testdeltager brugte slet ikke søgefunktionaliteten og de resterende 4 fandt først søgesiden
et stykke henne i testen. Én testdeltager fandt først søgesiden til allersidst i testen og
udtrykte derefter at hun var ked af at hun først fandt den så sent. På Figure \ref {fig:forside} kan man se hvor svært søgefeltet er at få øje på.

\begin{figure}[htbp]
    \centering
    \fbox{\includegraphics[width=1.2\textwidth]{Pics/forsidefaarup}}
    \caption{Forsiden med den svære at se søgefunktion.}
    \label{fig:forside}
\end{figure}

\item[\smallproblem]{\textbf{Søgefeltet på søgesiden har ikke fokus}}

2 deltagere forsøgte at skrive deres søgekriterier umiddelbart efter de var kommet ind på
søgesiden under den antagelse at søgefeltet havde fokus. Det krævede dog at de først klikkede
på søgefeltet.

\item[\seriousproblem]{\textbf{Søgefunktionen søger kun på faarupsommerland.dk}}

Søgefunktionen finder ingen resultater der befinder sig på faarupbooking.dk eller faarupshop.dk. Dette skabte problemer for en af vores deltagere der ville finde informationer om et bestemt hotel, og hun endte med at gå ind på hotellets egen hjemmeside.

\item[\good]{\textbf{Søgefunktionen fungerer}}

Når der blev søgt på emner der var på faarupsommerland.dk oplevede vi i langt størstedelen af tilfældende at deltagerne der brugte søgefeltet fandt siden de ledte efter i første forsøg.

\item[\smallproblem]{\textbf{Svært at komme tilbage til hovedsiden fra bookingportalen og webshoppen}}

Menupunkterne \emph{Overnatning} og \emph{E-shop} åbner et nyt vindue. Flere testdeltagere
opdagede ikke dette og det gav dem problemer da de prøvede at komme tilbage til hovedsiden.

For at komme tilbage til hovedsiden prøvede én testdeltager at klikke på \emph{Du er her:
Forside}-linket, men dette førte hende ikke til hovedsiden som forventet. Andre deltagere
prøvede at bruge Tilbage-knappen (hvilket ikke virkede da siden var åbnet et nyt vindue)
og én valgte at indtaste \url{faarupsommerland.dk} i addressefeltet.

\item[\smallproblem]{\textbf{Fårup Sommerland-logoet fører ikke altid til forsiden}}

På bookingportalen og webshoppen bliver man ikke ført til hovedsiden hvis man klikker på Fårup
Sommerland-logoet. Et par testdeltagere løb ind i dette problem når de prøvede at komme
tilbage til hovedsiden.

\item[\smallproblem] \textbf{\emph{Mad og drikke} og \emph{Butikker} er placeret forskelligt}

Disse to punkter minder rigtigt meget om hinanden i opbygning og hvilken information de indeholder. Alligevel er de placeret under forskellige menupunkter. \emph{Planlæg dit besøg} og \emph{Parken} henholdsvis. Dette forvirrede også flere deltagere der kiggede under butikker for at finde informationer om spisesteder.

\end{kommentarer}


\section{Booking}
\begin{kommentarer}
\item[\good] \textbf{Bestilling af overnatning}

Bestillingsboksen er nemt tilgængelig fra bookingportalens forside. Det er nemt at finde hvor man bestiller fra forsiden (Se Figure \ref{fig:forsidebooking}. 


\begin{figure}[htbp]
    \centering
    \fbox{\includegraphics[width=1.3\textwidth]{Pics/faarupbooking}}
    \caption{Faarupbooking.dk's forside. Her kan man tydeligt se hvor man skal skrive sin information for at booke et ophold.}
    \label{fig:forsidebooking}
\end{figure}

\item[\smallproblem] \textbf{Fejlmeddelelser}

Da en testdeltager glemte et felt under bestillingen af en overnatning gav siden en
fejlmeddelelse der ikke fortalte hvilken felter der ikke var udfyldt. Meddelelsen kan ses på
Figur~\ref{fig:fejlmeddelelsebooking}.

\begin{figure}[htbp]
    \centering
    \fbox{\includegraphics[width=0.6\textwidth]{Pics/faarupfejlbooking}}
    \caption{Fejlmeddelsen under booking når man mangler et eller flere felter.}
    \label{fig:fejlmeddelelsebooking}
\end{figure}

\item[\goodidea] \textbf{Det glemte hotel}

En af testdeltagerne prøvede at booke siden ved først at finde hotellet under
``Overnatningssteder''. Da hun trykkede på hotellet hun ville have tog det hende ind på
bookingsiden, hvor den havde glemt hvilket hotel hun havde valgt. Dette kunne spare tid hvis
den huskede hotellet.

\item[\seriousproblem] \textbf{Ingen søgning}

Flere af vores brugere forsøgte at bruge søgefunktionen på hovedsiden til at finde et
hotellet, denne søgening søger ikke på undersiderne. Derudover er der ingen søgfunktion på
undersiden.

\end{kommentarer}


\section{Information}

\begin{kommentarer}


\item[\good] \textbf{Meget information}

Flere deltagere nævnte at de følte at der var rigtigt meget relevant information på siden.

\item[\seriousproblem] \textbf{Meget tekst}

Alle deltagerene fandt en eller flere gange en side hvor de skulle til at læse mere tekst end de gad for at finde de informationer de ledte efter. Flere af deltagerene sagde at de ville have givet op og ringet til Fårup Sommerland i stedet. 

\item[\smallproblem] \textbf{\emph{Find vej} er mangelfuld}

Når brugerne fandt \emph{Find vej} var flere af dem skuffede over af at de blev mødt af en side med irrelevant tekst. En af dem blev irriteret på siden, fordi man stadig skal bruge en adresse eller koordinater hvis man har en GPS. Adressen er faktisk slet ikke at finde på \emph{Find vej}. Dog var der en der blev glad for kørselsbekrivelserne. 

\item[\smallproblem] \textbf{Svært at finde konkrete informationer om mad og drikke}

Under \emph{Mad og drikke} havde deltagerne svært ved at finde konkrete informationer. F.eks. om man selv måtte tage mad med, eller om der blev solgt pølser. Fårup Grillens side er det sted de fleste kiggede for information om pølser, men fik i stedet en masse information om hvordan grillen ikke er. De fleste missede det faktum at Grillen faktisk sælger hotdogs.

\item[\seriousproblem]{\textbf{``Information''-menupunkt er et sort-hul}}

% Dårlig formulering undskyld
Når deltagerne fik en opgave, hvor de skulle finde noget specifikt, og det ikke
var åbenlyst under hvilket et af menupunkterne de kunne finde det, klikkede de
på ``Information''. Det resulterede i at de klikkede meget rundt på siden.
Menupunktet ``Information'' er simpelthen for generelt, så lige meget hvad
deltageren leder efter, forestiller de sig at det måske kan være under
``Information''.

\end{kommentarer}


\section{Konsistens}

\begin{kommentarer}
\item[\smallproblem]{\textbf{Navn på E-shop}}

I menulinjen står der ``E-shop'', men andre steder på hjemmesiden nævnes det som
``webshoppen''. Der var ingen af vores deltagere der brugte e-shop-linket i
menubjælken, under den opgave hvor de skulle købe billetter.
\end{kommentarer}

\section{Forsiden}

\begin{kommentarer}
\item[\seriousproblem]{\textbf{Forsiden løser ingen vigtige opgaver}}

Der ikke én af vores deltagere der brugte forsiden til at løse opgaverne. De
vendte heller ikke tilbage til forsiden iløbet at testen, undtagen når de skulle
tilbage fra enten booking- eller e-shopsiden.

\item[\good]{\textbf{Forsiden er sjov}}

Da vi spurgte deltagerne til hvad de synes om forsiden, svarede de fleste
positivt. Den generelle mening var at de syntes at den var sjov og spændende.

\item[\smallproblem]{\textbf{Forsiden er for lang}}

En af vores deltagere klagede over at forsiden var så lang at man ikke kunne
have det hele på ét skærmbillede, og skulle scrolle ned af den for at få alle
informationerne.

\item[\goodidea]{\textbf{Få inspiration fra engelsk forside}}

Den engelske version af forsiden har en god indgangsvinkel til siden. Der er et
banner med ofte brugt information, så man slipper for at navigere rundt og lede
på siden.
\end{kommentarer}

\subsection{Egne kommentarer} % (fold)
\label{sub:Egne kommentarer forsiden}

% subsection Egne kommentarer (end)
\begin{kommentarer}
  \item[\seriousproblem]{\textbf{Skiftende banner bliver hvid}}

  Det skiftende banner på forsiden viser en hvid baggrund i tredje og fjerde
  billede. Hvis man klikker på den hvide baggrund, får man en fejlmeddelelse (Figur~\ref{fig:fejlmeddelelse}), hvor man kan se systeminformation.
\end{kommentarer}

\begin{figure}[htbp]
    \centering
    \fcolorbox{gray}{white}{\includegraphics[width=0.9\textwidth]{Pics/faarupfejlside2}}
    \caption{Denne applikationsfejl vises hvis man klikker på den hvide banner på forsiden.}
    \label{fig:fejlmeddelelse}
\end{figure}

\section{Udseende}
\begin{kommentarer}
  \item[\good]{\textbf{Skovtema}}
  
  Deltagerne syntes overvejende godt om det grønne skovtema, med
  undtagelse af deltager 4, som ikke var så vild med skovtemaet.

  \item[\smallproblem]{\textbf{Egern-animation}} 
  
  Tre af deltagerne kunne godt lide egernet i baggrundsbilledet. En deltager
  synes det var dumt, men at det måske passede til siden, og en anden synes rent
  ud at det var irriterende.
\end{kommentarer}

\subsection{Egne kommentarer} % (fold)
\label{sub:Egne kommentarer udseende}

\begin{kommentarer}
  \item[\smallproblem]{\textbf{Overskrift på undermenuer hvide}}

  Som set på Figure \ref{fig:hvideblokke} oplever vi på nogle af vores PC'er at overskriften på undermenuerne ikke blev
  vist, og at der i stedet var et hvidt felt, hvor overskriften skulle være,
  fordi overskriften var implementeret i flash.
\end{kommentarer}
% subsection Egne kommentarer (end)

\begin{figure}[htbp]
    \centering
    \fbox{\includegraphics[width=1.1\textwidth]{Pics/HvideBlokke}}
    \caption{Overskrifterne bliver til hvide blokke på nogle maskiner.}
    \label{fig:hvideblokke}
\end{figure}

\section{Andet}
\begin{kommentarer}

\item[\criticalproblem]{\textbf{E-Shop var lukket}}

Vi kunne ikke teste e-shoppen da den var lukket.

\item[\smallproblem]{\textbf{Funzone er gemt væk}}

Testdeltagerne kunne godt lide ideen om funzone, men den er gemt så langt væk de ikke ville opdage den hvis de ikke ledte efter den.

\item[\smallproblem]{\textbf{Højdeguiden er et screenshot}}

Højdeguiden er et screenshot af et Microsoft Excel ark. Dette er ikke optimalt. En af vores deltagere prøvede at lukke screenshottet og havde svært med at finde ud komme ud af den side.

\item[\smallproblem]{\textbf{404-side er ikke vejledende}}

Hvis man prøver på at gå ind på en side der ikke findes får man en 404-side der er meget sparsom på information. Desuden er søgefeltet fjernet på denne side, hvilket er beklageligt, da det kunne hjælpe folk hen til det de leder efter. I teksten kunne der også være et par generelle tips til hvordan man finder det man leder efter, via menuerne eller søgning. 

\end{kommentarer}



\clearpage
\appendix
\addcontentsline{toc}{section}{Appendiks}

\section{Fremgangsmåde}

Vi udførte alle test i løbet af to dage, hvor vi den første dag udførte to
tests, og tre tests den næste dag.

Vi udarbejdede drejebogen ved først at sætte os ned og tale om hvad vi ville forvente af faarupsommerland.dk, inden vi havde været inde og bruge siden selv. Derefter gik vi ind på siden og fandt ud af hvad siden egentlig kunne, og tilpassede vores opgaver efter dette. Vi testede opgaverne på en fra gruppen der ikke havde brugt hjemmesiden meget, for at finde åbenlyse fejl, og sørge for at opgaverne var naturlige. Derefter afleverede vi vores drejebog til instruktoren, og indarbejdede de rettelser vi fik tilbage.

Alle tests foregik præcis efter drejebogen (Appendiks \ref{apx:drejebog}), og tog mellem 45 og 60 minutter hver.
Vi overvejede at videooptage testene, men fravalgte dette da vi vuderede at arbejdsbyrden ville være alt for høj i forhold til hvor meget vi ville få ud af at se det igen. Vi optog lyden til et par af vores tests, men brugte det ikke til noget.
I løbet af vores tests, sørgede vi for at dedikere én person pr. test
til at skrive noter, en som tester og en til at holde styr på opgaverne.
Vi byttede roller fra test til test, og sørgede for at den person der
kendte deltagerne ikke testede dem han kendte i forvejen. Vi ville muligvis have kunnet få bedre testresultater ved at det altid var den samme der testede, tog noter eller håndterede opgaver, så man kunne vænne sig til hvordan det foregik, men vi syntes at det var vigtigere at alle fra gruppen prøvede alle rollerne, så alle havde en fornemmelse for hvordan det var.
Vi sørgede for at holde en pause efter hver test, hvor vi diskuterede resultaterne af den foregående test, samt skrev de vigtigste punkter ned. 

Efter testene mødtes vi og gennemgik test-noterne i fællesskab, og opstillede konkrete kommentarer ud fra disse noter. Disse kommentarer blev opdelt i kategorier og er udgangspunktet for vores rapport.


\section{Drejebog}
\label{apx:drejebog}

\paragraph{Startbetingelser}
\begin{opgaver}
\item Vi vil starte med at give en kort introduktion til deltageren, hvor vi
forklarer at dette er en test af hjemmesiden og ikke af deltageren, og hvordan
en tænke-højt-test foregår.
\item Deltageren skal bruge en Windows-datamat med Internet Explorer 8
installeret.
\item Internet Explorer vil være åbent, på Kongehusets hjemmeside.
\item Deltagerne får opgaverne udleveret på papir, én af gangen.
\end{opgaver}

\paragraph{Indledende spørgsmål}
\begin{opgaver}
\item Har du været i Fårup Sommerland før? Hvis ja, hvor ofte? Og hvad synes du
om det?
\item Har du været lignende steder for nylig? (Sommerland Sjælland, Bakken,
Bonbonland) Hvis ja, hvor ofte? Og hvad synes du om det?
\item I hvilken forbindelse har du sidst været et af disse steder?
\item Kan du beskrive din nærme familie? (antal, køn og alder)
\item Hvor ofte bruger du internettet, og til hvad?
\item Har du brugt Fårup Sommerlands hjemmeside før?
\item Har du brugt nogle af de lignende steders hjemmesider før?
\item Hvad er dine erfaringer med at bestille billetter eller ferieophold på
internettet?
\item I hvillken grad bruger du internettet til at undersøge et sted, før du
køber billet eller ophold?
\end{opgaver}

\newcommand{\note}[1]{\newline {\small\it Note: #1}}
\paragraph{Opgaver}
\begin{opgaver}
\item Find Fårup Sommerlands hjemmeside.
    \note{Kan deltageren finde siden?}

\item {\it Åbent spørgsmål}

    En af dine venner fortalte for nylig i begejstrede vendinger om et
    besøg i Fårup Sommerland. Hvis du skulle besøge Fårup Sommerland, hvilke
    oplysninger har du brug for? Find disse oplysninger på Fårup Sommerlands
    hjemmeside.
    \note{Kan brugeren finde frem til de informationer som hun finder nødvendige?}

\item {\it Åbningstider/sæson}

    Du overvejer at tage til Fårup Sommerland. Find ud af hvornår de har
    åbent og bestem en passende dato. Hvad er åbningstiderne på den dag du har
    valgt?
    \note{Åbningstiderne er en essentiel oplysning på hjemmesiden. \\ Åbningstiderne for de forskellige dele af sæsonen findes under menupunktet ``Tider og Priser'', i bunden af siden.}

\item {\it Priser}
    \begin{opgaver}
    \item Find billetprisen hvis du skulle tage din nærmeste familie med.
    \note{Fårup Sommelands største målgruppe er børnefamilier. \\ Billetpriser findes under menupunktet ``Tider og Priser''}
    \item Hvor meget koster billetter der også har mad inkluderet?  
    \note{For at finde disse billetter skal man gå ind under menupunktet ``Tider og Priser'', og vælge ``Fårup Inclusive'' i menuen i vestre side.}
    \end{opgaver}

\item {\it Bestil billetter}

    Bestil billetter til Fårup Sommerland for din familie næste weekend.
    \note{Leder siden brugeren videre til webshoppen? \\ Det er pt. ikke muligt at bestille billetter i webshoppen.}

\item {\it Adresse/rutevejledning}

    Find Fårup Sommerlands adresse og find ud af hvordan du nemmest
    kommer dertil.
    \note{Fårup Sommerland ligger midt ude i en skov. Derfor er der vigtigt at kunne finde vej. \\ Findes under ``Planlæg dit besøg'' $\rightarrow$ ``Find vej'' $\rightarrow$ ``Kørselsbeskrivelser'' $\rightarrow$ ``Find den korteste rute her'' $\rightarrow$ brug Kraks ruteplanlægger.}

\item {\it Sovemuligheder/spisemuligheder}
    \begin{opgaver}
    \item Du har besluttet dig for at dig og din familie gerne vil bruge 2 dage
    i Fårup Sommerland. Find prisen for et ophold med overnatning på et hotel
    startende den 20. december.
    \note{Fårup Sommerland ønsker at siden fører brugerne naturligt over til bookingsiden. \\ Man skal vælge ``Overnatning'', udfylde ``Bestil din ferie her''-boksen, og så scroll'e ned eller vælge hotel i højre side.}
    \item Du ved et af dine familiemedlemmer elsker pølser, og ikke kan undvære det i en familietur. Kan man få pølser i Fårup Sommerland?
    \note{Kan man finde informationer om parkens faciliteter? \\ Der er både grill-selv og Fårup-grill.}
    \end{opgaver}

\item {\it Forlystelser/attraktioner}

    Din nevø vil meget gerne prøve rutsjebanen Lynet i Fårup Sommerland.
    Hvor høj skal man være for at måtte prøve forlystelsen?
    \note{Kan man finde informationer om parkens forlystelser? \\ Man skal være 120 cm.}

\item {\it Find billetter/bedste tilbud}

    En kollega fra dit arbejde vil tage en
    4-dages tur til Fårup Sommerland med hans kone og to børn. De har selv et
    sommerhus de vil bo i. Hjælp ham med at finde ud af hvad det vil koste.
    \note{Kan brugerne finde ud af bruge Fårups special billetter? \\ Almindelig billetter kan findes under ``Tider og Priser''. Miniferie-kort ville passe godt, men findes kun under booking og ingen klar pris kan nævnes.}

\item {\it Se betingelser for ophold}

    Du vil gerne booke et ophold for en uge på et af hotellerne, men du
    er ikke sikker på om du kan den dato. Find ud af hvor længe inden opholdsdatoen
    du kan få dine penge tilbage.
    \note{Forståelige betingelser kan være med tila t øge sidens troværdighed. \\ >30 dage inden: 300 kr. gebyr; 30--8 dage inden: 50\% af prisen; <8 dage inden: ingen godtgørelse.}

\item {\it Se reviews af opholdssteder}

    Du har hørt fra venner at Blokhus Klit Golf er et godt hotel. Find,
    om muligt, en beskrivelse og bedømmelser af dette hotel.
    \note{Kan brugerne finde denne funktion, og ville de bruge den? \\ Under ``Overnatning'' $\rightarrow$ ``Overnatningssteder'' $\rightarrow$ ``Blokhus Klit Golf'' $\rightarrow$ ``Se bedømmelser''}

\item {\it Booking af ophold}
\note{Ingen af disse opgaver kan løses, da vi ikke har mulighed for at foretage en booking hos Fårup Sommerland}
    \begin{opgaver}
    \item Bestil hotel-overnatning og billetter til weekendturen med din
    familie på dette hotel.
    \item Dit arbejde har spurgt dig om hvornår du kommer hjem fra ferie, og du
    har glemt hvilken slutdato du har givet hotellet. Find slutdatoen på din
    booking.
    \item Du har lånt en campingvogn af en ven, og ønsker at bo i denne under
    jeres ophold. Ændrer din reservation til at i kan bo på en campingplads.
    \item Du har fundet ud af at du havde andre planer den weekend. Annullér
    reservationen eller skift datoen til weekenden efter.
    \end{opgaver}

\item {\it Sæsonkort}

    Du har været så glad for Fårup Sommerland at du gerne vil have et
    sæsonkort. Find ud af hvilke goder et sæsonkort giver, og find priser for
    disse.
    \note{For at finde disse billetter skal man gå ind under menupunktet ``Tider og Priser'', og vælge ``Sæsonkort'' i menuen i vestre side.}

\item {\it Diverse}
    \begin{opgaver}
    \item Har Fårup Sommerland nogle ledige stillinger?
    \note{Mange unge arbejder i Fårup sommeren over. \\ Pt. ingen ledige stillinger.}
    \item Må man medbringe sit eget mad og drikke til Fårup Sommerland?
    \note{Det må man godt.}
    \item Børnene der skal med på turen begynder at blive meget utålmodige, og
    de har brug for noget underholdning. Har Fårup Sommerland noget
    børneunderholdning på hjemmesiden?
    \note{God børneunderholdning kan hjælpe til at sælge siden til børnene. \\ Under ``Parken'' kan man vælge ``Funzone'', hvor der er underholdning for børn.}
    \end{opgaver}
\end{opgaver}

\paragraph{Eftersnak}
\begin{opgaver}
\item Hvad var dit indtryk af siden?
\item Nævn 3 gode og 3 dårlige ting ved siden.
\item Var opgaverne realistiske? Nogen af opgaverne du ikke kunne sætte dig ind i?
\item Havde du nemt/svært ved at finde de ting du skulle på siden?
\item Hvad synes du om menu-punkterne og opdelingen af siden?
\item Var du i tvivl om hvor du skulle kigge først for at løse opgaverne, eller fandt du det nemt?
\item Fandt du siden troværdig? Ville du have tiltro til at bruge faarupbooking.dk eller faarupshoppen.dk? Hvad kunne gøre den mere troværdig?
\item Hvad synes du om forsiden?
\item Fandt du forsiden hjælpsom til løsning af opgaverne?
\item Hvad synes du om skov-temaet på siden, og den animerede baggrund?
\item Ville du have lyst til at bruge siden igen?
\item Hvad kunne få dig til at besøge siden regelmæssigt?
\end{opgaver}

\newpage

\section{Forsiden}

\begin{figure}%
\centering
\subfloat[][\textbf{Forsiden på Kolding Kommunes websted}. Nogle testdeltagerne syntes ikke, at denne side lignede en rigtig forside.]{
\includegraphics[width=\textwidth]{Pics/frontpage}
}%
\qquad
\subfloat[][\textbf{Websiden Kommunen}. Nogle testdeltagere troede, at denne side var webstedets forside, fordi den ``ligner'' en forside, og fordi linket Kommunen i venstremenuen altid er nemt tilgængeligt.]{
\includegraphics[width=\textwidth]{Pics/kommune}
}
\caption{Here are the first two figures of a continued figure.}%
\label{fig:cont}%
\end{figure}

\begin{table}[!ht]
\centering
\begin{tabular}{p{0.1\textwidth}p{0.8\textwidth}}
& \\
\vspc\smallproblem& \textbf{De direkte link til webstedets forside er usynlige}.\\
& Ingen testdeltagere vidste, at de kunne komme til forsiden ved at klikke på kommunens segl i øverste venstre hjørne, selv om dette er en udbredt konvention. Kun nogle få testdeltagere fik øje på linket \emph{Forside} i menulinien øverst i billedet.\newline

Det havde den konsekvens, at mange testdeltagere opfattede siden \emph{Kommunen} som forsiden, fordi linket til denne side er meget synligt (øverst i menuen i venstre side). Nogle testdeltagere brugte Back-knappen for at komme tilbage til forsiden og andre valgte at indtaste URL'en.\\
& \\
\vspc\smallproblem & \textbf{Topmenuen er usynlig}.\\
& Kun nogle få testdeltagere så topmenuen. De testdeltagere, som fik øje på den, kritiserede funktionerne for at være mindre relevante, bortset fra \emph{Forside}.\newline

Testleders kommentar: Topmenuen går i ét med billedfrisen. Separér dem.
Flyt f.eks. følgende funktioner til topmenuen: \emph{Forside}, \emph{Vigtige telefonnumre}, \emph{Åbningstider}, \emph{Søg}, \emph{Genveje}, \emph{Indeks} (dvs. sitemap).
\end{tabular}
\end{table}%

\clearpage

\section{Testdeltagernes opgaveløsning}

\renewcommand\pic[1]{\includegraphics[trim=0 5 0 0,scale=0.8]{Pics/#1}}
\begin{table*}[!htp]
\begin{center}
\label{tbl: oversigt}
\small
\begin{tabular}{lccccc}
    \toprule
    Opgave\textbackslash deltager  & 1       & 2     & 3     & 4             & 5 \\
    \midrule
    1. Find webstedet              & \good & \good & \good & \good       & \good    \\[2mm]
    2. Åbent spørgsmål             & \good   & \good & \good & \smallproblem & \smallproblem    \\[2mm]
    3. Find åbningstider           & \filler & \good & \good & \filler       & \seriousproblem  \\[2mm]
    4.a. Find priser               & \filler & \good & \good & \filler       & \good            \\[2mm]
    4.b. Find priser med mad       & \filler & \good & \good & \filler       & \criticalproblem \\[2mm]
    5. Bestil billetter            & \filler & \good & \good & \filler       & \criticalproblem \\[2mm]
    6. Adresse og rutevejledning   & \filler & \good & \good & \filler       & \criticalproblem \\[2mm]
    7.a. Find pris for hotel       & \filler & \good & \good & \filler       & \criticalproblem \\[2mm]
    7.b. Kan man få pølser         & \filler & \good & \good & \filler       & \criticalproblem \\[2mm]
    8. Minimumshøjde for Lynet     & \filler & \good & \good & \filler       & \criticalproblem \\[2mm]
    9. Billetter til kollega       & \filler & \good & \good & \filler       & \criticalproblem \\[2mm]
    10. Afbestillinsbetingelser    & \filler & \good & \good & \filler       & \criticalproblem \\[2mm]
    11. Se reviews                 & \filler & \good & \good & \filler       & \criticalproblem \\[2mm]
    12.a. Bestil hotel-overnatning & \filler & \good & \good & \filler       & \criticalproblem \\[2mm]
    12.b. Slutdato for ophold      & \filler & \good & \good & \filler       & \criticalproblem \\[2mm]
    12.c. Ændre reservation        & \filler & \good & \good & \filler       & \criticalproblem \\[2mm]
    12.d. Anullér reservation      & \filler & \good & \good & \filler       & \criticalproblem \\[2mm]
    13. Pris for sæsonkort         & \filler & \good & \good & \filler       & \criticalproblem \\[2mm]
    14.a. Ledige stillinger        & \filler & \good & \good & \filler       & \criticalproblem \\[2mm]
    14.b. Egen mad og drikke       & \filler & \good & \good & \filler       & \criticalproblem \\[2mm]
    14.c. Underholdning            & \filler & \good & \good & \filler       & \criticalproblem \\[2mm]
    \bottomrule
\end{tabular}
\caption{Oversigt over testdeltagernes opgaveløsning.}
\end{center}
\end{table*}

\clearpage

\section{Inspektion af websted}
\label{apx:Inspektion af websted}

Vi lagde hurtigt mærke til at faarupsommerland.dk, faarupbooking.dk og
faarupshop.dk havde fuldstændig samme design, indgik som undersider i
faarupsommerland.dk, men hver var deres egen side. Det synes vi er
forvirrende og inkonsistent.

404-siden er ikke hjælpsom. I stedet for at forklare, at siden ikke er
blevet fundet i systemet, er fejlmeddelsesen "Siden kan ikke vises", hvilket er
misvisende. Samtidig ville det her være oplagt at give brugeren mulighed for at
søge, eller give en oversigt over flere navigationspunkter der kunne have
interesse, og det er ikke gjort.

Hvis man indtaster forkert information på bookingsiden, glemmer siden al
indtastet information, og man skal begynde forfra. Det er heller ikke muligt at
vælge 0 overnatninger, hvis man kun ønsker billetter, og der er heller ikke
nogen forklaring på, hvorfor man ikke kan. Samtidig bliver miniferie-kort
ikke nævnt under "Tider og priser", men kun under "Booking" og
"Parken"->"Family".

Ved nogle links åbner faarupbooking.dk i en ny fane, og ved andre links
åbner den på samme side, hvilket er inkonsistent.

\section{Sammenligning med inspektion}
\label{apx:sammenligning-inspektion}

Når vi sammenligner vores egen initielle inspektion med vores resultater fra
brugertesten, står det klart at hvor vores inspektion tager fat i ting, som vi
fra et teknisk perspektiv synes er inkongruente eller "forkerte", så var
testresultaterne meget mere fokuserede på brugerens opgaver, og om det
lykkedes eller ej.

Altså var vores inspektion detaljeorienteret, mens brugertestene var
opgave-orienteret.

\section{Vores kommentarer}
\label{apx:vores-kommentarer}


\end{document} 
