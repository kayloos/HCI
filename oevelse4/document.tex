\documentclass[a4paper, 12pt]{article}

% rubber: module bibtex
% rubber: depend ../litteratur.bib

\usepackage[utf8]{inputenc}
\usepackage[T1]{fontenc}
\usepackage[danish]{babel}
\usepackage[bitstream-charter]{mathdesign}
\usepackage{graphicx}
%% \usepackage{float}
%% \usepackage{varioref}
\usepackage{booktabs}
\usepackage[footnotesize,margin=1cm]{caption}
\usepackage{enumitem}
\usepackage{subfig}
\usepackage{url}
\usepackage[draft]{fixme}

\newcommand{\oevelsenr}{4}
\newcommand{\oevelsetitel}{PACT-analyse med interview}
\newcommand{\afleveringsdato}{6. oktober 2011}

\usepackage{fancyhdr}
\fancyhf{}
\fancyhead[L]{\footnotesize Gruppe 61}
\fancyhead[R]{\footnotesize Øvelse \oevelsenr}
\pagestyle{fancy}

\usepackage[style=numeric]{biblatex}
\addbibresource{../litteratur.bib}

\setdescription{
  font=\normalfont\itshape,
  style=sameline,
  leftmargin=\parindent
}

\begin{document}

%% \begin{titlepage}
\begin{center}

{\large Menneske-datamaskine interaktion}

\vspace{2cm}

\rule{\linewidth}{0.5mm}\\[3mm]

{\LARGE Øvelse \oevelsenr}\\[6mm]
{\Huge \oevelsetitel}

\rule{\linewidth}{0.5mm}

\vfill

{\bf Gruppe 61}\\[3mm]
\begin{tabular}{lr}
  Claus Skou Nielsen & \texttt{ptf992} \\
  Kasper Passov & \texttt{pvx884} \\
  Michael Budde & \texttt{skx295} \\
  Niels Ørbæk Christensen & \texttt{cxr861}
\end{tabular}

\vspace{1cm}

{\bf Afleveringsdato}\\[3mm]
\afleveringsdato

\vfill

\end{center}
\end{titlepage}


\newpage
\setcounter{page}{1}
\fancyfoot[C]{\thepage}

\section{Fremgangsmåde}
\label{sec:Fremgangsmaade}

Vi har fulgt forslaget om fremgangsmåden som den bliver beskrevet i
opgaveformuleringen. Vi lagde ud med at udføre en tjekliste, som vi kunne bruge
senere til at interviewe brugerne.

Derefter fandt vi på fire kerneopgaver, der dækkede det vigtigste af sidens
funktionalitet. Det ledte os hen til at udføre en prototype, således at alle
kerneopgaverne er dækket af prototypen. Vi valgte at lave en prototype af papir,
da det er langt det letteste at udarbejde i den indledende fase af design.

Desuden sørgede vi for at overholde reglerne for brugsvenligt design, som det er
blevet gennemgået ved forelæsning, samt vores egne forestillinger om, hvad der
er brugervenligt, og hvad der ikke er.

Vi afleverede vores tjekliste, prototype og kerneopgaver til vores instruktor,
og reviderede vores opgave i forhold til de modtagne rettelser.

Dernæst satte vi aftaler op med interviewdeltagere, og foretog løbende
interviews i et interval på fire dage. Vi sørgede for at tage noter under hvert
interview, og vælge interviewdeltagere i forskellige målgrupper.

\section{Interviewresultater}
\label{sec:Interviewresultater}

\section{Kerneopgaver}
\label{sec:Kerneopgaver}
\begin{itemize}
\item Se reviews af resturanter i et område
\item Se reviews af restauranter for et bestemt køkken
\item Find bestemt restaurant og se reviews/kommentare, stjerner
\item Skriv review/kommentar for en bestemt restaurant
\end{itemize}

\section{Målgruppebeskrivelser}
\label{sec:Maalgruppebeskrivelser}

\begin{itemize}
  \item Familier med små børn
  \begin{itemize}
    \item En familie der alle bor sammen og tager ud at spise i fælleskab. De har yngre børn og det er derfor vigtigt for dem at stemningen er god og at det er nemt at komme hen til restauranten.
  \end{itemize}
  \item 25-30 årige storby-par uden børn.
  \begin{itemize}
    \item Unge mennesker der har taget en længerevarende uddannelse og har fået gode job og derfor har en sund økonomi og overskud i hverdagen. De har mange kulturelle interesser og vil gerne prøve nye populære restauranter.
  \end{itemize}
  \item Pensionister
 \begin{itemize}
    \item Ældre mennesker der nyder at bruge deres pensionsår på at gå ud på restauranter og lede efter gode tilbud. De har tid til at planlægge deres restaurant-besøg nøje og går efter at få meget værdi for pengene.
  \end{itemize}
  \item Karrieremenneske
 \begin{itemize}
    \item Mennesker i de sene 30'ere til de tidlige 50'ere der arbejder meget og har deres karriere som en top-prioritet. De har luft i økonomien, men har ikke megen tid til at lede efter den rigtige restaurant som de skal bruge til arbejdsfrokoster eller lignende.
  \end{itemize}
  \item Studerende
 \begin{itemize}
    \item Unge personer i 20'erne der uddanner sig i eller omkring
      København. De har ikke særligt mange penge at gøre godt med, og
      pris er derfor vigtigt for dem. De har desuden intet motoriseret
      køretøj og ønsker at finde restauranter i nærheden.
  \end{itemize}
\end{itemize}

\subsection{Uddybelse af målgruppen \emph{Karrieremenneske}}
Det karriereorienterede menneske går ofte på restuarant, både som privatperson eller i forretningsøjemed. Personen kan både være mand eller kvinde, men vil oftest være 30-50 år, og have en god indtjening der giver mulighed for at spise ofte på restaurant uden at skulle tænke over prisen.

Det karriere orienterede menneske ønsker at finde gode restauranter hurtigt. Det er vigtigt at restauranterne er gode og præsentable, da de også tit skal bruges til forretningsfrokoster. Derfor forventer hun at man nemt kan se anmeldelser og beskrivelser af restuarenterne, som hun vil bruge til at vudere om de passer til den givne frokost. 

Denne slags bruger vil næste udelukkende benytte sig af at søge og finde restauranter, fremfor at skrive anmeldelser af restauranter og kommentarer af andres restauranter. Det er fordi hun ikke ser restuarantbesøgene som ren fornøjelse. Hendes restaurantbesøg har som regel et formål; enten at underholde en forretningspartner, eller bare at få et måltid mad når hun ikke selv har tid/overskud til at lave mad. 

Disse brugere vil næsten altid have ret meget erfaring med internettet, som er de eneste forudsætninger for at kunne bruge siden effektivt. De vil også bruge lignende sider til at effektivere andre dele af deres hverdag, som transport, udlandsrejser eller indkøb.




\section{Personas}
\label{sec:Personas}

\subsection{Uddybelse af målgruppen \emph{Karrieremenneske}}

\section{Scenarier}
\label{sec:Scenarier}

\section{Testresultater fra prototype}
\label{sec:Testresultater fra prototype}

\section{Erfaringer}
\label{sec:Erfaringer}

\clearpage
\appendix

\section*{Tjekliste}

Du vil komme til at prøve en tidlig test-udgave (prototype) som ikke vil ligne det endelige site.
\begin{itemize}
\item Vi tester websitet og konceptet, ikke dig. Det er ikke en
  eksamen. Vi sætter pris på alle dine kommentarer.
\item Hvad er dine internet-færdigheder? Bruger du internettet ofte?
  Har du selv et website?
\end{itemize}

\paragraph{Anledning}
\begin{itemize}
\item Hvor tit spiser du ude?
\item I hvilke anledninger spiser du ude?
\end{itemize}

\paragraph{Brugsmønstre}
\begin{itemize}
\item Hvordan vælger du steder at spise?
\item Hvad er vigtigt når du vælger en restaurant at spise på?
  \begin{itemize}
  \item Type, kvalitet, pris, lokation eller stemning?
  \end{itemize}
\item Hvor meget betyder anmeldelser fra f. eks. en avis eller hjemmeside?
\item Hvor meget betyder anmeldelser fra andre brugere?
\item Hvad forventer du at en restaurationsvurderingshjemmeside indeholder?
\item Hvor vigtig er kommunikation med andre brugere af hjemmesiden?
\item Hvad lægger du vægt på i en anmeldelse?
\item Hvad ville få dig til at skrive en personlig anmeldelse af en restaurant?
\item Hvad ville kunne få dig til at registrere dig som bruger på hjemmesiden?
Rabatter, anbefalinger eller andet?
\item Ville du gøre brug af at kunne tilkendegive din vurdering af en anmeldelse
og hvorfor?
\end{itemize}

\paragraph{Debriefing}
\begin{itemize}
\item Hvad synes du om webstedet?
\item Hvilke 2--3 ting fungerer bedst på webstedet?
\item Hvilke 2--3 ting trænger mest til at blive forbedret?
\end{itemize}

\printbibliography[heading=bibnumbered,title=Litteraturliste]

\end{document}
