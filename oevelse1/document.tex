\documentclass[10pt]{article}

\usepackage[utf8]{inputenc}
\usepackage[T1]{fontenc}

\usepackage[english]{babel}
\usepackage[bitstream-charter]{mathdesign}

\usepackage{fancyhdr}
\pagestyle{fancy}
\fancyhead[L]{Menneske-datamaskine Interaktion\\ Gruppe 61}
\fancyhead[R]{2011/09/08\\ Øvelse \#1}


\begin{document}

\section*{Målelige krav til brugervenlighed af P-automat}

\begin{enumerate}
\item Automaten skal kunne findes på 30 sekunder.
\item Automaten må ikke være mere end 30 sekunder væk.

\item Brugeren skal let kunne finde ud af hvordan kortet skal vende.
\item Brugeren skal kunne se, hvilke mønter der kan bruges.

\item 90\% skal kunne købe en billet i første forsøg.
\item 80\% skal være i stand til at aflæse priser.
\item Det skal være tydeligt at se pris og udløbstidspunkt inden billeten
  købes.
\item Automaten skal kunne gøre brugeren opmærksom på at brugeren har
  glemt kortet.

\item Billetten skal tydeligt fremvise udløbstidspunkt, varighed, beløb.
\item 80\% af ikke dansk-talende talende skal kunne benytte automaten.
\item 99\% skal kunne købe en billet med den ønskede varighed.
\item 90\% førstegangsbrugere skal være tilfredse med brugeroplevelsen.
\item 95\% erfarne brugere skal være tilfredse med brugeroplevelsen.
\end{enumerate}

\section*{Evaluering}

\begin{enumerate}
\item {\it Automaten skal kunne findes på 30 sekunder.}

  Automaten var nem at se, fra alle steder i parkeringsområdet, da den
  stod steder hvor biler ikke kunne blokere for den. 

\item {\it Automaten må ikke være mere end 30 sekunder væk.}

  I parkeringsområdet var der rigeligt med p-automater på begge sider af
  vejen, så man var aldrig mere end 30 meter væk fra en automat.

\item {\it Brugeren skal let kunne finde ud af hvordan kortet skal vende.}

  Der var en illustrering af et kort der vendte med chippen opad som
  skulle vise at det var denne vej at kortet skulle vende. Der var dog
  ikke vist hvilken vej magnetstriben skulle vende. Hvis man indsatte
  kortet på en forkert led, gav den umiddelbart ingen fejlbesked, men
  begyndte i stedet transaktionen. Da det fejlede meldte den fejl i
  transaktionen, men intet nærmere om hvor fejlen befandt sig. Desuden
  fulgte der med fejlen adskillige høje bib-lyde der ikke just hjalp på
  forvirringen.   

\item {\it Brugeren skal kunne se, hvilke mønter der kan bruges.}
  Der var fint markeret med billeder hvilke mønter der kunne bruges. Vi
  har dog ikke testet mønt-indkastet.

\item {\it 90\% skal kunne købe en billet i første forsøg.}

  I vores sparsomme forsøg lykkedes det for 0\% at købe en billet i
  første forsøg, men det skyldtes at vi med vilje saboterede processen
  for at fremprovokere fejl.

\item {\it 80\% skal være i stand til at aflæse priser.}

  Bortset fra nogle forkortelser af ugedagene var det meget nemt af
  aflæse priser. 

\item {\it Det skal være tydeligt at se pris og udløbstidspunkt inden billeten
  købes.}

  Det var vidst tydeligt på automatens skærm hvornår billetten galdt
  til, og hvor mange penge man havde valgt. 

\item {\it Automaten skal kunne gøre brugeren opmærksom på at brugeren har
  glemt kortet.}

  I tilfælde af at man glemmer sit kort viser skærmen en meddelse der
  lyder "Fjern Kort", men går ikke yderligere opmærksom på at kortet
  ikke er taget ud. Dette er ikke optimalt, da billetten bliver
  udstedt inden man har fjernet kortet, og denne skal tages fra en
  skuffe længere nede på automaten. Skærmen er placeret i toppen af
  automaten, og man vil derfor nemt kunne overse denne
  meddelse. Automaten fik tidligere gjort det meget klart at den kan
  sige lyde, så en mulig løsning kunne være også at bruge en lyd til
  at gøre brugeren opmærksom på meddelsen. En anden løsning ville være
  at billetten ikke blev udstedt før at kortet var fjernet, som man
  også kan opleve i andre billetautomater. 

\item {\it Billetten skal tydeligt fremvise udløbstidspunkt, varighed, beløb.}

  Billetten viste udløbstidspunkt og beløb (uden gebyr). Den viste ikke
  varighed, men viste til gengæld også hvornår billetten var udstedt. 

\item {\it 80\% af ikke dansk-talende talende skal kunne benytte automaten.}

  Vi havde ikke nogle ikke engelsk-talende testpersoner, men automaten
  har en knap med forskellige flag på, der understøttede 5 forskellige
  sprog. Dog ikke japansk, som faktisk er den 3. største turistgruppe i
  Danmark.
  

\item {\it 99\% skal kunne købe en billet med den ønskede varighed.}

  100\% af vores testpersoner fik en billet med den ønskede varighed. Man
  kan ikke vælge hvor lang tid billetten skal vare, og så betale præcis
  det, men det er nemt at finde den rigtige varighed ved at prøve
  sig frem.

\item {\it 90\% førstegangsbrugere skal være tilfredse med brugeroplevelsen.}

  Brugeroplevelsen var god når man ikke lavede fejl, men fejlmeddelserne
  og de høje lyde udgjorde et noget stressende element i fejlfindings
  processen.

\item {\it 95\% erfarne brugere skal være tilfredse med brugeroplevelsen.}

  Vi har ikke kunnet teste dette, men som der også står ovenfor er
  brugeroplevelsen betydeligt bedre når man ikke laver fejl, og chancen
  for disse fejl mindskes med erfaringen.
\end{enumerate}

\end{document}
